% \iffalse
%
% fixme.dtx --- Doc file for the FiXme package (code and documentation)
%
% Copyright (C) 1998 Didier Verna.
%
% PRCS: $Id: fixme.dtx 1.3 Tue, 05 Jan 1999 17:49:34 +0100 verna $
%
% Author:        Didier Verna <verna@inf.enst.fr>
% Maintainer:    Didier Verna <verna@inf.enst.fr>
% Created:       Thu Dec 10 16:04:01 1998 under XEmacs 21.2 (beta 6)
% Last Revision: Thu Dec 10 16:04:01 1998
%
% This file is part of FiXme.
%
% FiXme is distributed in the hope that it will be useful,
% but WITHOUT ANY WARRANTY; without even the implied warranty of
% MERCHANTABILITY or FITNESS FOR A PARTICULAR PURPOSE.
%
% You are NOT ALLOWED to distribute this file alone. However,
% you are allowed to distribute this file under the condition
% that it is distributed along with the whole FiXme package.
% If you receive only some of these files from someone, complain!
%
% You are NOT ALLOWED to modify this file, or any other file in the
% FiXme package, neither to distribute modified versions of it. However,
% you are allowed to freely incorporate (possibly modified) parts of
% this file in other files with clearly different names, provided
% that the original authors are given full credit for their work.
%
% You are NOT ALLOWED to charge a fee for distribution or use of this
% package or any derivative work as described above, other than for the
% physical act of transferring copies. 
%
% Commentary:
%
% Contents management by FCM version 0.1.
%
% Code:
%
%<*driver>
\documentclass[a4paper]{ltxdoc}
\usepackage[draft,warning]{fixme}
\begin{document}
% \OnlyDescription
\DocInput{fixme.dtx}
\end{document} 
%</driver>
%
% \fi
%
% ^^A $Format: "% \\newcommand{\\packageversion}{$PackageVersion$}"$
% \newcommand{\packageversion}{1.0-b4}
% ^^A $Format: "% \\newcommand{\\packagedate}{$PackageDate$}"$
% \newcommand{\packagedate}{1999/01/05}
% \MakeShortVerb{\|}
% \date{\today}
% \title{\textsf{FiXme} -- a \LaTeX{} package fort inserting FIXME notes
% in draft documents \thanks{This document describes \textsf{FiXme} version
% \packageversion , last updated \packagedate .}}
% \author{Didier Verna\\
% |mailto:verna@inf.enst.fr|\\
% |http://www.inf.enst.fr/~verna/|}
% \maketitle
%
%
% \begin{abstract}
% In the process of writing a long document, it is common to let some parts
% unwritten or uncomplete, and come back to them later. When you have an
% important ammount of parts to be {\em fixed} in your document, you probably
% want clues about where such parts are left (which I call ``fixme notes'').
% That's what this package does.\\ 
% The \textsf{fixme} package is copyright \copyright\ 1998 - Didier Verna. All
% rights reserved.
% \end{abstract}
%
%
% \section{Introduction}
% The \textsf{fixme} package essentially provides one macro, |\fixme{}|, that
% allows you to stick clues about parts which should be fixed in your
% document. If your are in {\em final} mode, any fixme notes left in the
% document generates an error. If you are in {\em draft} mode, then the
% behavior of the |\fixme{}| macro is controlled by a set of options, which
% let you output warnings, insert margin notes\fixme{Here's a sample fixme
% note for you guys}, add entries in the index etc.
%
% 
% \section{User Interface}
% To use the package, simply say |\usepackage[|\meta{options}|]{fixme}| in the
% preamble of your document. The available options are described below.
%
% \subsection{Available Options}
% First, you have to decide whether you're in {\em draft} or {\em final}
% mode \footnote{Notice that those options are usually given to
% \texttt{\bslash documentclass\{\}}, which in turn passes them to all
% packages.}. Don't use both options, they are mutually exclusive.
% \par
% \DescribeEnv{final}\DescribeEnv{draft} In {\em final} mode (the default)
% any remaining fixme note in the document generates an error. Well, if there
% are parts to be fixed, the document is not final right? On the contrary, the
% {\em draft} option tells \textsf{fixme} to issue fixme notes in a manner
% defined by the following options (those ones can be mixed): 
% \par
% \DescribeEnv{nomargin} By default, \texttt{fixme} inserts the fixme notes
% in the margin, like the one shown in the introduction of this document.
% This is the original utility I wanted to implement. The note appears in
% small characters, flushed to the right, and prefixed with a bold
% \textbf{FIXME: }. If you want to disable this feature, pass the {\em
% nomargin} option to the package.
% \par
% \DescribeEnv{index} With the {\em index} option, fixme notes can also be 
% inserted in the index of your document. All \texttt{fixme} index entries
% appear under the \textbf{***FIXME} key at the beginning of the index.
% \par
% \DescribeEnv{warning}\DescribeEnv{info} Additionally, you might want to
% output warnings on the terminal or in the log file. The {\em warning} option 
% outputs a warning on the terminal with the fixme note and the line number in
% from source file. This warning is also registered in the log file. If you
% only want to register the fixme notes in the log file, use the {\em info}
% option instead. It does essentially the same, but doesn't mess up your
% terminal output.
% \begin{quote}
% - And what if I want output on the terminal, but not in the log file?\\
% - Well, just get lost \{:-)
% \end{quote}
% 
% \StopEventually{\par Well, I think that's it. Enjoy using FiXme!!
%   \vfill\hfill\small Copyright \copyright 1998 -- Didier Verna}
% \subsection{Available Macros}
% \DescribeMacro{\fixme} This is the one and only macro that you will use to
% generate a fixme note in your document. It takes one argument: the note to
% generate. You have an example of such a note in the margin of the
% introduction, which was generated by the command
% |\fixme{Here's a sample fixme note for you guys}|.
% \par
% \DescribeMacro{\FiXmeMargin}\DescribeMacro{\FiXmeIndex}
% \DescribeMacro{\FiXmeWarning}\DescribeMacro{\FiXmeInfo}
% Those macros (one per formatting option) are considered internal. They are
% used by the |\fixme{}| command to generate the fixme note in different ways.
% However, if you are not satisfied with the way they perform their job, you
% can redefine them using |\renewcommand{}| if you like. Please refer to the
% next section to see how they are implemented.
%
%
% \section{The Code}
% For the interested reader, here is a short sequential description of the
% code implementing this package. First, the package identifies itself:
%    \begin{macrocode}
\NeedsTeXFormat{LaTeX2e}[1994/06/01]
%% $Format: "\\ProvidesPackage{fixme}[$PackageDate$]"$
\ProvidesPackage{fixme}[1999/01/05]

%    \end{macrocode}
% Then, we define our initial commands. Internal commands are all empty,
% except for |\FiXmeMargin{}| which is the default. Also, the initial behavior 
% of |\fixme{}| corresponds to the {\em final} mode (generating an error).
%    \begin{macrocode}
\newcommand{\FiXmeMargin}[1]{%
\marginpar{\footnotesize\flushright\textbf{FIXME: }\em #1}}
\newcommand{\FiXmeWarning}[1]{}
\newcommand{\FiXmeInfo}[1]{}
\newcommand{\FiXmeIndex}[1]{}

\newcommand{\fixme}[1]{%
\PackageError{fixme}{`FIXME: #1' left in final version.}}

%    \end{macrocode}
% Now, we declare our options. First, the options controlling which actions
% should be performed when a \verb!\fixme{}! command is issued,
%    \begin{macrocode}
\DeclareOption{nomargin}{\renewcommand{\FiXmeMargin}[1]{}}

\DeclareOption{warning}{\renewcommand{\FiXmeWarning}[1]{%
\PackageWarning{fixme}{`FIXME: #1'}}}

\DeclareOption{info}{\renewcommand{\FiXmeInfo}[1]{%
\PackageInfo{fixme}{`FIXME: #1'}}}

\DeclareOption{index}{\renewcommand{\FiXmeIndex}[1]{%
\index{***@FIXME:!#1}}}

%    \end{macrocode}
% and then the global behavior options:
%    \begin{macrocode}
\DeclareOption{final}{}
\DeclareOption{draft}{\renewcommand{\fixme}[1]{%
\FiXmeMargin{#1}%
\FiXmeWarning{#1}%
\FiXmeInfo{#1}%
\FiXmeIndex{#1}}}

%    \end{macrocode}
% Finally, we process the options.
%    \begin{macrocode}
\ProcessOptions

%    \end{macrocode}
%
% \Finale\PrintChanges
