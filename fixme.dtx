% \iffalse                                                 -*- mode: LaTeX -*-
%
% fixme.dtx --- Doc file for the FiXme package (code and documentation)
%
% Copyright (C) 1998-2002 Didier Verna.
%
% PRCS: $Id: fixme.dtx 1.53 Tue, 23 Apr 2002 17:58:20 +0200 didier $
%
% Author:        Didier Verna <didier@lrde.epita.fr>
% Maintainer:    Didier Verna <didier@lrde.epita.fr>
% Created:       Thu Dec 10 16:04:01 1998
% Last Revision: Tue Mar 26 11:02:22 2002
%
% This file is part of FiXme.
%
% FiXme may be distributed and/or modified under the
% conditions of the LaTeX Project Public License, either version 1.1
% of this license or (at your option) any later version.
% The latest version of this license is in
% http://www.latex-project.org/lppl.txt
% and version 1.1 or later is part of all distributions of LaTeX
% version 1999/06/01 or later.
%
% FiXme consists of the files listed in the file `README'.
%
%
% Commentary:
%
% Contents management by FCM version 0.1.
%
%
% Code:
%
%<*driver>
\documentclass[a4paper]{ltxdoc}
% \OnlyDescription
% \CodelineIndex
% \RecordChanges
\begin{document}
\DocInput{fixme.dtx}
\end{document}
%</driver>
%
% \fi
%
% \catcode`\�=14
% \CheckSum{636}
%% \CharacterTable
%%  {Upper-case    \A\B\C\D\E\F\G\H\I\J\K\L\M\N\O\P\Q\R\S\T\U\V\W\X\Y\Z
%%   Lower-case    \a\b\c\d\e\f\g\h\i\j\k\l\m\n\o\p\q\r\s\t\u\v\w\x\y\z
%%   Digits        \0\1\2\3\4\5\6\7\8\9
%%   Exclamation   \!     Double quote  \"     Hash (number) \#
%%   Dollar        \$     Percent       \%     Ampersand     \&
%%   Acute accent  \'     Left paren    \(     Right paren   \)
%%   Asterisk      \*     Plus          \+     Comma         \,
%%   Minus         \-     Point         \.     Solidus       \/
%%   Colon         \:     Semicolon     \;     Less than     \<
%%   Equals        \=     Greater than  \>     Question mark \?
%%   Commercial at \@     Left bracket  \[     Backslash     \\
%%   Right bracket \]     Circumflex    \^     Underscore    \_
%%   Grave accent  \`     Left brace    \{     Vertical bar  \|
%%   Right brace   \}     Tilde         \~}
%
% ^^A $Format: "% \\newcommand\\version{v$Version$}"$
% \newcommand\version{v2.0}
% ^^A $Format: "% \\newcommand\\releasedate{$ReleaseDate$}"$
% \newcommand\releasedate{2002/04/03}
% ^^A $Format: "% \\newcommand\\packagecopyright{$LaTeXCopyright$}"$
% \newcommand\packagecopyright{Copyright \copyright{} 1998--2002 Didier Verna}
% \newcommand\fixme{\textsf{FiXme}}
% \newcommand\auctex{AUC-\TeX}
% \MakeShortVerb{\|}
% \date\today
% \title{\fixme{} -- a \LaTeXe{} package for inserting fixme notes
% in your documents \thanks{This document describes \fixme{}
% \version, release date \releasedate.}}
% \author{Didier Verna\\
% \texttt{mailto:didier@lrde.epita.fr}\\
% \texttt{http://www.lrde.epita.fr/\~{}didier/}}
% \maketitle
%
%
% \begin{abstract}
% In the process of writing a long document, it is a common practice to leave
% some parts unwritten or uncomplete, and come back to them later. In such
% cases, you probably want to stick clues about which parts need to be
% ``fixed'', where they are located, and what needs to be done. This is what I
% call ``fixme notes''. The purpose of this package is to provide you with
% convenient ways to insert \fixme{} notes in your documents.\par
% The \fixme{} package is \packagecopyright{}, and distributed under the terms
% of the LPPL license.
% \end{abstract}
%
%
% \section{Description}
% \fixme{} notes can appear in several forms, depending on which option(s) you
% pass to the package. These forms currently are: inline, marginal notes,
% footnotes and index entries (the notes are also logged and some of them are
% displayed on the terminal). You can use multiple forms (possibly all) at the
% same time. Additionally, you can summarize all \fixme{} notes in a list
% similar to the ``list of figures'' for instance.\par
% All these forms are available when you're working in \texttt{draft} mode.
% In \texttt{final} mode, the behavior is slightly different: any remaining
% critical note generates an error (the compilation aborts), while non
% critical ones are just removed from the document's body (they're still
% logged though).
%
%
% \section{User Interface}
% To use the package, simply say |\usepackage[|\meta{options}|]{fixme}| in the
% preamble of your document.
%
% \subsection{Inserting notes}
% \DescribeMacro{\fixme}
% The main command for inserting a \fixme{} note in your document is the
% |\fixme| macro. It takes the note to insert as its mandatory argument. Notes
% inserted via this command are considered fatal to your document's
% processing (see the next section).\par
% \DescribeMacro{\fxnote}
% \DescribeMacro{\fxwarning}
% \DescribeMacro{\fxerror}
% As of version 2.0, \fixme{} provides three new macros that insert
% meta-comments about the document, rather than real fixmes. These
% comments have three different importance levels: note, warning and error.
% The corresponding macros obey the same syntax as |\fixme|. However, none
% of the notes inserted via these macros are fatal.
%
% \subsection{Controlling the behavior of \fixme}
% \DescribeEnv{final}
% \DescribeEnv{draft}
% The global behavior of the package is controlled via the two standard
% options \texttt{final} and \texttt{draft}. These options are usually given
% to |\documentclass{}| which in turn passes them to all packages.\par
% In \texttt{draft} mode, the notes are logged (in the log file, that is) and
% appear in the text as specified (please refer to the ``layout'' sections).
% Additionally, warnings, errors and fatal errors are also displayed on the
% terminal.\par
% In final mode, non fatal notes (those generated by the |\fx*| commands) are
% still logged, but they're removed from the document's body. On the other
% hand, any remaining fatal note (generated by |\fixme|) will abort
% compilation with an informative message. This will help you track down
% forgotten important caveats in your document. Let me say it again: final
% documents can only have notes, warnings, and (non fatal) errors left. Well,
% if, for some reason, you rilly rilly want to compile in \texttt{final} mode
% with critical \fixme{} notes left behind, you always have the possibility to
% pass the \texttt{draft} option to \fixme{} directly\ldots\par
% The \texttt{final} mode has been chosen as the default because \LaTeXe{}
% itself behaves this way.\par
%
% \subsection{Controlling the layout}
% \DescribeEnv{inline}
% \DescribeEnv{margin}
% \DescribeEnv{footnote}
% \DescribeEnv{index}
% \fixme{} notes can appear in several forms (that can be combined) in your
% document. The forms currently supported are: inline (directly in the text),
% marginal notes, footnotes, and index entries. To activate a particular
% layout, use the corresponding option. By default, only the \texttt{margin}
% layout is active.\par
% Each layout option has a counterpart that deactivates it. The counterpart
% option has the same name, prefixed with \texttt{no}. For instance, if you
% don't want marginal notes, use the \texttt{nomargin} option.\par
% \DescribeMacro{\listoffixmes}
% \fixme{} remembers where you put \fixme{} notes in a toc-like file whose
% extension is \texttt{lox}. The |\listoffixmes| macro generates the list of
% all \fixme{} notes in a manner similar to that of the ``list of figures''
% for instance. A standard layout is automatically used for the `article',
% `report', `book' classes and their koma-script replacements. If another
% class is used, the `article' layout is selected. Also, note that if no
% \fixme{} note remain in the document, this macro doesn't generate an empty
% list, but rather stays silent. It also stays silent in \texttt{final} mode,
% regardless of the presence of remaining notes.
%
% \subsection{Customizing the layout}
% The \texttt{inline}, \texttt{margin} and \texttt{footnote} layouts have two
% parts: a ``prefix'' which depends on the note level, and the note
% itself. The prefix is one of ``\fixme{} note:'', ``\fixme{} warning:'',
% ``\fixme{} error:''  or simply ``\fixme{}:'', and appears in bold. The note
% itself appears emphasized.
% \DescribeMacro{\FXInline}
% \DescribeMacro{\FXMargin}
% \DescribeMacro{\FXFootnote}
% These layouts are implemented thanks to the corresponding |\FX*| macros that
% you can redefine if you whish. Each such macro takes two mandatory
% arguments: the prefix and the note, in that order.\par
% \DescribeMacro{\FXIndex}
% The |\FXIndex| macro is used for the \texttt{index} layout. All \fixme{}
% index entries appear under the ``\fixme{}'' key in the symbols section.
% There are 4 subcategories under this key, as many as there are note levels.
% By default, only the first 3 of them are used though (fatal errors do not
% appear under a subkey, but directly under the \fixme{} key). The notes are
% numbered in the index.\par
% \DescribeMacro{\FXNote}
% \DescribeMacro{\FXWarning}
% \DescribeMacro{\FXError}
% \DescribeMacro{\FXFatal}
% These commands are the ones used by |\fxnote|, |\fxwarning|, |\fxerror| and
% |\fixme| to log (and optionally display on the terminal) the notes. They
% take the note itself as mandatory argument.
%
% \subsection{Fancy fruit salad layout}
% \DescribeEnv{user}
% \DescribeMacro{\FXUser}
% And what if you wanted something rilly, rilly fancy? I mean, like playing
% ``A Candle in the Wind'' on your loudspeaker and displaying a picture of
% Pamela Anderson naked on your boss' computer screen? Well, you \textbf{can}
% do that. Just pass the \texttt{user} option to the package (it also has its
% \texttt{nouser} counterpart), and define an |\FXUser| macro in the following
% manner:\\
% |\newcommand{\FXUser}[2]{|\meta{fancy fruit salad layout job}|}|\\ The
% arguments are the prefix, and the note itself, in that order.
%
% \subsection{Selecting a language}
% \DescribeEnv{english}
% \DescribeEnv{french}
% \DescribeEnv{francais}
% \DescribeEnv{spanish}
% \DescribeEnv{italian}
% \DescribeEnv{german}
% \DescribeEnv{ngerman}
% \fixme{} currently supports English, French, Spanish, Italian and German.
% You can select the language you want to use by passing the corresponding
% option (these options are usually given directly to |\documentclass{}| which
% in turn passes them to all packages). The \texttt{french} and
% \texttt{francais} options are synonyms. The \texttt{german} and
% \texttt{ngerman} options are currently equivalent.\par
% If you want a finer grain on the language-dependent parts of \fixme{}, the
% following macros are provided and can be redefined.\par
% \DescribeMacro{\fixmenoteprefix}
% \DescribeMacro{\fixmewarningprefix}
% \DescribeMacro{\fixmeerrorprefix}
% \DescribeMacro{\fixmefatalprefix}
% \DescribeMacro{\fixmelogo}
% The |\fixme*prefix| macros define the prefix for the four different note
% levels. They make intensive use of the macro |\fixmelogo| ;-)\par
% \DescribeMacro{\fixmeindexname}
% \DescribeMacro{\fixmenoteindexname}
% \DescribeMacro{\fixmewarningindexname}
% \DescribeMacro{\fixmeerrorindexname}
% \DescribeMacro{\fixmefatalindexname}
% The macro |\fixmeindexname| defines the main \fixme{} index key. The other
% ones define the different index subkeys for each note level. Please note that
% an empty name for a subkey means that you don't actually want a subcategory
% (that's the case by default for fatal errors). The corresponding notes will
% then appear directly under the main \fixme{} key. For that reason, a non
% empty subkey must end with an exclamation mark.\par
% \DescribeMacro{\listfixmename}
% |\listfixmename| defines the title for the ``list of fixmes'' section.
%
% \subsection{Counters}
% \DescribeMacro{\thefixmecount}
% \DescribeMacro{\thefixmenotecount}
% \DescribeMacro{\thefixmewarningcount}
% \DescribeMacro{\thefixmeerrorcount}
% \DescribeMacro{\thefixmefatalcount}
% As of \fixme{} version 2.0, four counters are provided. They count
% respectively the total number of \fixme{} notes, and the number of
% level-specific ones. At the end of processing, a \fixme{} usage summary is
% displayed and logged.
%
%
% \section{\auctex{} support}
% \auctex{} is a powerful major mode for editing \TeX{} documents in
% \textsf{Emacs} or \textsf{XEmacs}. In particular, it provides automatic
% completion of macro names once they are known. \fixme{} supports \auctex{}
% by providing a style file named \texttt{fixme.el} which contains \auctex{}
% definitions for the relevant macros. This file should be installed to a
% location where \auctex{} can find it (usually in a subdirectory of your
% \LaTeX{} styles directory). Please refer to the \auctex{} documentation for
% more information on this.
%
% \section{Changes}
% \begin{itemize}
% \item[v2.0] New feature: note levels\\
% New feature: \fixme{} note counters and usage summary\\
% Suggestions from Kasper B. Graversen |<kbg@dkik.dk>|\\
% Support for Spanish thanks to Agust\'in Mart\'in |<agusmb@netscape.net>|
% \item[v1.5] New appearance option: \texttt{inline}.
% \item[v1.4] Support for the koma-script classes\\
% Fix bug reported by Ulf Jaenicke-Roessler
% |<ujr@physik.phy.tu-dresden.de>|: the |\listoffixmes| command didn't work
% when called before the first \fixme{} note.
% \item[v1.3] Support for Italian thanks to Riccardo Murri |<murri@phc.unipi.it>|
% \item[v1.2] Support for German thanks to Harald Harders |<h.harders@tu-bs.de>|
% \end{itemize}
%
% \StopEventually{\par Well, I think that's it. Enjoy using \fixme{}!
%   \vfill\hfill\small \packagecopyright{}.}
%
% \section{The Code}
% \begin{macro}{\fixmelogo}
%   Here's the preamble:
% \end{macro}
%    \begin{macrocode}
\NeedsTeXFormat{LaTeX2e}
� $Format: "\\ProvidesPackage{fixme}[$ReleaseDate$ v$Version$"$
\ProvidesPackage{fixme}[2002/04/03 v2.0
                        Insert fixme notes in your documents]

\newcommand*\fixmelogo{\textsf{FiXme}}

%    \end{macrocode}
% \subsection{List macros}
% \begin{macro}{\listoffixmes}
%   \begin{macro}{\listoffixmes@final}
%     \begin{macro}{\listoffixmes@draft}
%       |\listoffixmes| is empty by default. The |final| or |draft| options
%       will provide the proper definition. Lists are output in a document
%       class dependant fashion. Classes currently recognized are article,
%       report, book and their koma-script replacements.
%
% ^^A \changes{v1.4}{2001/09/04}{Fix bug reported by Ulf Jaenicke-Roessler}
%       In order to prevent the List of Fixme's chapter heading from being
%       generated when there are no fixmes, a test on the existence of the lox
%       file is performed. There's a slight bug left however: after removing
%       the last fixme, one ends up with an empty lox file, so the heading
%       still appear. Previously, this was done by checking if some fixmes
%       were given, but that was buggy: the list of fixme's could not appear
%       before the first fixme note\ldots
%     \end{macro}
%   \end{macro}
% \end{macro}
%    \begin{macrocode}
\newcommand\listoffixmes{}
\def\listoffixmes@final{}
\def\listoffixmes@draft{%
  \IfFileExists{\jobname .lox}{\@listoffixmes@pretoc}{}%
  \@starttoc{lox}%
  \IfFileExists{\jobname .lox}{\@listoffixmes@posttoc}{}%
  }

%    \end{macrocode}
% \begin{macro}{\listfixmename}
%   In order to redefine the list title, users have to call |\renewcommand|.
%   I don't like this much: it would be cleaner to provide a real command
%   that redefines the title internally. However, I'll stick with the
%   current scheme because it is consistent with the rest of \LaTeXe's lists.
% \end{macro}
%    \begin{macrocode}
\newcommand*\listfixmename{}

%    \end{macrocode}
%
% \subsubsection{Standard classes}
% \begin{macro}{\listoffixmes@pretoc@article}
%   \begin{macro}{\listoffixmes@posttoc@article}
%     \begin{macro}{\l@fixme@article}
%       Let's start with the article version:
%     \end{macro}
%   \end{macro}
% \end{macro}
%    \begin{macrocode}
\def\@listoffixmes@pretoc@article{%
  \section*{\listfixmename%
    \@mkboth{\MakeUppercase\listfixmename}{\MakeUppercase\listfixmename}}%
  }
\def\@listoffixmes@posttoc@article{}
\def\l@fixme@article{\@dottedtocline{1}{1.5em}{2.3em}}

%    \end{macrocode}
% \begin{macro}{\listoffixmes@pretoc@report}
%   \begin{macro}{\listoffixmes@posttoc@report}
%     \begin{macro}{\l@fixme@report}
%       Here's the report version:
%     \end{macro}
%   \end{macro}
% \end{macro}
%    \begin{macrocode}
\def\@listoffixmes@pretoc@report{%
  \if@twocolumn
    \@restonecoltrue\onecolumn
  \else
    \@restonecolfalse
  \fi
  \chapter*{\listfixmename%
    \@mkboth{\MakeUppercase\listfixmename}{\MakeUppercase\listfixmename}}%
  }
\def\@listoffixmes@posttoc@report{\if@restonecol\twocolumn\fi}
\def\l@fixme@report{\@dottedtocline{1}{1.5em}{2.3em}}

%    \end{macrocode}
% \begin{macro}{\listoffixmes@pretoc@book}
%   \begin{macro}{\listoffixmes@posttoc@book}
%     \begin{macro}{\l@fixme@book}
%       And here's the book version:
%     \end{macro}
%   \end{macro}
% \end{macro}
%    \begin{macrocode}
\def\@listoffixmes@pretoc@book{%
  \if@twocolumn
    \@restonecoltrue\onecolumn
  \else
    \@restonecolfalse
  \fi
  \chapter*{\listfixmename%
    \@mkboth{\MakeUppercase\listfixmename}{\MakeUppercase\listfixmename}}%
  }
\def\@listoffixmes@posttoc@book{\if@restonecol\twocolumn\fi}
\def\l@fixme@book{\@dottedtocline{1}{1.5em}{2.3em}}

%    \end{macrocode}
%
% \subsubsection{koma-script classes}
% \begin{macro}{\listoffixmes@pretoc@scrartcl}
%   \begin{macro}{\listoffixmes@posttoc@scrartcl}
%     \begin{macro}{\l@fixme@scrartcl}
%       Let's start with the scrartcl version:
%     \end{macro}
%   \end{macro}
% \end{macro}
%    \begin{macrocode}
\def\@listoffixmes@pretoc@scrartcl{%
  \section*{\listfixmename%
    \@mkboth{\listfixmename}{\listfixmename}}%
  }
\def\@listoffixmes@posttoc@scrartcl{}
\def\l@fixme@scrartcl{\@dottedtocline{1}{1.5em}{2.3em}}

%    \end{macrocode}
% \begin{macro}{\listoffixmes@pretoc@scrreprt}
%   \begin{macro}{\listoffixmes@posttoc@scrreprt}
%     \begin{macro}{\l@fixme@scrreprt}
%       Here's the scrreprt version:
%     \end{macro}
%   \end{macro}
% \end{macro}
%    \begin{macrocode}
\def\@listoffixmes@pretoc@scrreprt{%
  \if@twocolumn
    \@restonecoltrue\onecolumn
  \else
    \@restonecolfalse
  \fi
  \chapter*{\listfixmename%
    \@mkboth{\listfixmename}{\listfixmename}}%
  }
\def\@listoffixmes@posttoc@scrreprt{\if@restonecol\twocolumn\fi}
\def\l@fixme@scrreprt{\@dottedtocline{1}{1.5em}{2.3em}}

%    \end{macrocode}
% \begin{macro}{\listoffixmes@pretoc@scrbook}
%   \begin{macro}{\listoffixmes@posttoc@scrbook}
%     \begin{macro}{\l@fixme@scrbook}
%       And here's the scrbook version:
%     \end{macro}
%   \end{macro}
% \end{macro}
%    \begin{macrocode}
\def\@listoffixmes@pretoc@scrbook{%
  \if@twocolumn
    \@restonecoltrue\onecolumn
  \else
    \@restonecolfalse
  \fi
  \chapter*{\listfixmename%
    \@mkboth{\listfixmename}{\listfixmename}}%
  }
\def\@listoffixmes@posttoc@scrbook{\if@restonecol\twocolumn\fi}
\def\l@fixme@scrbook{\@dottedtocline{1}{1.5em}{2.3em}}

%    \end{macrocode}
%
% \subsection{Layout macros}
% ^^A \changes{v1.5}{2002/02/18}{New appearance option: \texttt{inline}}
% \begin{macro}{\FXInline}
%   \begin{macro}{\FXMargin}
%     \begin{macro}{\FXFootnote}
%       The macros to implement different layouts for the fixme notes:
%     \end{macro}
%   \end{macro}
% \end{macro}
%    \begin{macrocode}
\newcommand\FXInline[2]{%
  \textbf{#1}: \emph{#2}}
\newcommand\FXMargin[2]{%
  \marginpar{\footnotesize\flushright\textbf{#1}: \emph{#2}}}
\newcommand\FXFootnote[2]{%
  \footnote{\textbf{#1}: \emph{#2}}}

%    \end{macrocode}
% \begin{macro}{\fixmeindexname}
%   \begin{macro}{\FXIndex}
%     The macros for index generation:
%   \end{macro}
% \end{macro}
%    \begin{macrocode}
\newcommand*\fixmeindexname{}
\newcommand\FXIndex[1]{%
  \index{***@\fixmeindexname:!#1}}

%    \end{macrocode}
% \begin{macro}{\FXNote}
%   \begin{macro}{\FXWarning}
%     \begin{macro}{\FXerror}
%       \begin{macro}{\FXFatal}
%         And the ones for logging and terminal output:
%       \end{macro}
%     \end{macro}
%   \end{macro}
% \end{macro}
%    \begin{macrocode}
\newcommand\FXNote[1]{%
  \GenericInfo{%
    (FiXme)\@spaces\@spaces\@spaces\@spaces}{%
    Package FiXme Note: `#1'}}
\newcommand\FXWarning[1]{%
  \PackageWarning{FiXme}{`#1'}}
\newcommand\FXError[1]{%
  \GenericWarning{%
    (FiXme)\@spaces\@spaces\@spaces\@spaces}{%
    Package FiXme Error: `#1'}}
\newcommand\FXFatal[1]{%
  \GenericWarning{%
    (FiXme)\@spaces\@spaces\@spaces\@spaces}{%
    Package FiXme: `#1'}}

%    \end{macrocode}
%
% \begin{macro}{\fixmefatalprefix}
% \begin{macro}{\FiXmeInline}
%   \begin{macro}{\FiXmeMargin}
%     \begin{macro}{\FiXmeFootnote}
%       \begin{macro}{\FiXmeUser}
%         \begin{macro}{\FiXmeIndex}
%           \begin{macro}{\FiXmeInfo}
%             \begin{macro}{\FiXmeWarning}
%               Apart from |\fixmefatalprefix|, the following macros are kept
%               only to preserve backward compatibility. They are used only by
%               critical notes and are made to call the new versions by
%               default:
%             \end{macro}
%           \end{macro}
%         \end{macro}
%       \end{macro}
%     \end{macro}
%   \end{macro}
% \end{macro}
% \end{macro}
%    \begin{macrocode}
\newcommand*\fixmefatalprefix{}
\newcommand\FiXmeInline[1]{\FXInline{\fixmefatalprefix}{#1}}
\newcommand\FiXmeMargin[1]{\FXMargin{\fixmefatalprefix}{#1}}
\newcommand\FiXmeFootnote[1]{\FXFootnote{\fixmefatalprefix}{#1}}
\newcommand\FiXmeUser[1]{}
\let\FiXmeUser\empty

\newcommand\FiXmeIndex[1]{\FXIndex{#1}}
\newcommand\FiXmeInfo[1]{}
\newcommand\FiXmeWarning[1]{}

%    \end{macrocode}
% Booleans that control which layouts are selected:
%    \begin{macrocode}
\newif\iffixme@inline\fixme@inlinefalse
\newif\iffixme@margin\fixme@marginfalse
\newif\iffixme@footnote\fixme@footnotefalse
\newif\iffixme@user\fixme@userfalse
\newif\iffixme@index\fixme@indexfalse

%    \end{macrocode}
% \begin{macro}{\fixme@text}
%   This macro implements the common (textual) part of all notes:
% \end{macro}
%    \begin{macrocode}
\def\fixme@text#1#2{%
  \iffixme@inline\FXInline{#1}{#2}\fi%
  \iffixme@margin\FXMargin{#1}{#2}\fi%
  \iffixme@footnote\FXFootnote{#1}{#2}\fi%
  \iffixme@user\FXUser{#1}{#2}\fi}

%    \end{macrocode}
% Options that control which layouts are selected:
%    \begin{macrocode}
\DeclareOption{inline}{\fixme@inlinetrue}
\DeclareOption{noinline}{\fixme@inlinefalse}
\DeclareOption{margin}{\fixme@margintrue}
\DeclareOption{nomargin}{\fixme@marginfalse}
\DeclareOption{footnote}{\fixme@footnotetrue}
\DeclareOption{nofootnote}{\fixme@footnotefalse}
\DeclareOption{user}{\fixme@usertrue}
\DeclareOption{nouser}{\fixme@userfalse}
\DeclareOption{index}{\fixme@indextrue}
\DeclareOption{noindex}{\fixme@indexfalse}

%    \end{macrocode}
% The following options are here for backward compatibility, but are not used
% anymore:
%    \begin{macrocode}
\DeclareOption{info}{}
\DeclareOption{noinfo}{}
\DeclareOption{warning}{}
\DeclareOption{nowarning}{}

%    \end{macrocode}
%
% \subsection{\fixme{} notes}
% Here are the counters:
% \DescribeMacro{\thefixmecount}
% \DescribeMacro{\thefixmenotecount}
% \DescribeMacro{\thefixmewarningcount}
% \DescribeMacro{\thefixmeerrorcount}
% \DescribeMacro{\thefixmefatalcount}
%    \begin{macrocode}
\newcounter{fixmecount}
\newcounter{fixmenotecount}
\newcounter{fixmewarningcount}
\newcounter{fixmeerrorcount}
\newcounter{fixmefatalcount}

%    \end{macrocode}
% \begin{macro}{\fixmenoteprefix}
%   \begin{macro}{\fixmenoteindexname}
%     \begin{macro}{\fixme@note@draft}
%       \begin{macro}{\fixme@note@final}
%         The different flavors of fixme notes. Proper definitions will be
%         selected by the |draft| or |final| options. Let's start with the
%         note notes:
%       \end{macro}
%     \end{macro}
%   \end{macro}
% \end{macro}
%    \begin{macrocode}
\newcommand*\fixmenoteprefix{}
\newcommand*\fixmenoteindexname{}
\def\fixme@note@draft#1{%
  \fixme@text{\fixmenoteprefix}{#1}%
  \iffixme@index\FXIndex{\fixmenoteindexname\thefixmenotecount: #1}\fi%
  \FXNote{#1}%
  \addcontentsline{lox}{fixme}{\fixmenoteprefix: #1}}
\def\fixme@note@final#1{%
  \FXNote{#1}%
  \addcontentsline{lox}{fixme}{\fixmenoteprefix: #1}}

%    \end{macrocode}
% \begin{macro}{\fixmewarningprefix}
%   \begin{macro}{\fixmewarningindexname}
%     \begin{macro}{\fixme@warning@draft}
%       \begin{macro}{\fixme@warning@final}
%         Here are the warnings notes:
%       \end{macro}
%     \end{macro}
%   \end{macro}
% \end{macro}
%    \begin{macrocode}
\newcommand*\fixmewarningprefix{}
\newcommand*\fixmewarningindexname{}
\def\fixme@warning@draft#1{%
  \fixme@text{\fixmewarningprefix}{#1}%
  \iffixme@index\FXIndex{\fixmewarningindexname\thefixmewarningcount: #1}\fi%
  \FXWarning{#1}%
  \addcontentsline{lox}{fixme}{\fixmewarningprefix: #1}}
\def\fixme@warning@final#1{%
  \FXWarning{#1}%
  \addcontentsline{lox}{fixme}{\fixmewarningprefix: #1}}

%    \end{macrocode}
% \begin{macro}{\fixmeerrorprefix}
%   \begin{macro}{\fixmeerrorindexname}
%     \begin{macro}{\fixme@error@draft}
%       \begin{macro}{\fixme@error@final}
%         Here are the errors notes:
%       \end{macro}
%     \end{macro}
%   \end{macro}
% \end{macro}
%    \begin{macrocode}
\newcommand*\fixmeerrorprefix{}
\newcommand*\fixmeerrorindexname{}
\def\fixme@error@draft#1{%
  \fixme@text{\fixmeerrorprefix}{#1}%
  \iffixme@index\FXIndex{\fixmeerrorindexname\thefixmeerrorcount: #1}\fi%
  \FXError{#1}%
  \addcontentsline{lox}{fixme}{\fixmeerrorprefix: #1}}
\def\fixme@error@final#1{%
  \FXError{#1}%
  \addcontentsline{lox}{fixme}{\fixmeerrorprefix: #1}}

%    \end{macrocode}
% \begin{macro}{\fixmefatalindexname}
%   \begin{macro}{\fixme@fatal@draft}
%     \begin{macro}{\fixme@fatal@final}
%       And here are the fatal ones:
%     \end{macro}
%   \end{macro}
% \end{macro}
%    \begin{macrocode}
\newcommand*\fixmefatalindexname{}
\def\fixme@fatal@draft#1{%
  \iffixme@inline\FiXmeInline{#1}\fi%
  \iffixme@margin\FiXmeMargin{#1}\fi%
  \iffixme@footnote\FiXmeFootnote{#1}\fi%
  \iffixme@user\FiXmeUser{#1}\fi%
  \iffixme@index\FiXmeIndex{%
    \fixmefatalindexname\thefixmefatalcount: #1}\fi%
  \FXFatal{#1}%
  \addcontentsline{lox}{fixme}{\fixmefatalprefix: #1}}
\def\fixme@fatal@final#1{%
  \PackageError{FiXme}{`#1' fixme left in final version}{%
    You are processing your document in final mode,\MessageBreak
    but you still have some critical FiXme's left behind.\MessageBreak
    Type X to quit, fix your document, and rerun LaTeX.}}

%    \end{macrocode}
% \begin{macro}{\@fixme}
%   \begin{macro}{\fxnote}
%     \begin{macro}{\fxwarning}
%       \begin{macro}{\fxerror}
%         \begin{macro}{\fixme}
%           Here are the notes insertion macros. The different versions are
%           |\let| to the draft of final flavors defined above by the
%           \texttt{draft} or \texttt{final} options.
%         \end{macro}
%       \end{macro}
%     \end{macro}
%   \end{macro}
% \end{macro}
%    \begin{macrocode}
\def\@fixme#1#2{%
  \stepcounter{fixmecount}%
  \stepcounter{fixme#1count}%
  \csname fixme@#1\endcsname{#2}}

\newcommand\fxnote[1]{\@fixme{note}{#1}}
\newcommand\fxwarning[1]{\@fixme{warning}{#1}}
\newcommand\fxerror[1]{\@fixme{error}{#1}}
\newcommand\fixme[1]{\@fixme{fatal}{#1}}

%    \end{macrocode}
% \subsection{Language processing}
% Options that control which language is selected:
%    \begin{macrocode}
\DeclareOption{english}{%
  \renewcommand*\fixmenoteprefix{\fixmelogo~Note}
  \renewcommand*\fixmewarningprefix{\fixmelogo~Warning}
  \renewcommand*\fixmeerrorprefix{\fixmelogo~Error}
  \renewcommand*\fixmefatalprefix{\fixmelogo}
  \renewcommand*\fixmeindexname{\fixmelogo}
  \renewcommand*\fixmenoteindexname{**a@Notes:!}
  \renewcommand*\fixmewarningindexname{**b@Warnings:!}
  \renewcommand*\fixmeerrorindexname{**c@Errors:!}
  \renewcommand*\fixmefatalindexname{}
  \renewcommand*{\listfixmename}{List of Corrections}}
\DeclareOption{french}{%
  \renewcommand*\fixmenoteprefix{\fixmelogo~Note}
  \renewcommand*\fixmewarningprefix{\fixmelogo~Attention}
  \renewcommand*\fixmeerrorprefix{\fixmelogo~Erreur}
  \renewcommand*\fixmefatalprefix{\fixmelogo}
  \renewcommand*\fixmeindexname{\fixmelogo}
  \renewcommand*\fixmenoteindexname{**a@Notes:!}
  \renewcommand*\fixmewarningindexname{**b@Avertissements:!}
  \renewcommand*\fixmeerrorindexname{**c@Erreurs:!}
  \renewcommand*\fixmefatalindexname{}
  \renewcommand*{\listfixmename}{Liste des Corrections}}
\DeclareOption{francais}{\ExecuteOptions{french}}
\DeclareOption{spanish}{%
  \renewcommand*\fixmenoteprefix{\fixmelogo~Nota}
  \renewcommand*\fixmewarningprefix{\fixmelogo~Aviso}
  \renewcommand*\fixmeerrorprefix{\fixmelogo~Error}
  \renewcommand*\fixmefatalprefix{\fixmelogo}
  \renewcommand*\fixmeindexname{\fixmelogo}
  \renewcommand*\fixmenoteindexname{**a@Notas:!}
  \renewcommand*\fixmewarningindexname{**b@Avisos:!}
  \renewcommand*\fixmeerrorindexname{**c@Errores:!}
  \renewcommand*\fixmefatalindexname{}
  \renewcommand*{\listfixmename}{Lista de Correcciones}}
\DeclareOption{italian}{%
  \renewcommand*\fixmenoteprefix{\fixmelogo~Nota}
  \renewcommand*\fixmewarningprefix{\fixmelogo~Avviso}
  \renewcommand*\fixmeerrorprefix{\fixmelogo~Errore}
  \renewcommand*\fixmefatalprefix{\fixmelogo}
  \renewcommand*\fixmeindexname{\fixmelogo}
  \renewcommand*\fixmenoteindexname{**a@Note:!}
  \renewcommand*\fixmewarningindexname{**b@Avvisi:!}
  \renewcommand*\fixmeerrorindexname{**c@Errori:!}
  \renewcommand*\fixmefatalindexname{}
  \renewcommand*{\listfixmename}{Corrigenda}}
\DeclareOption{german}{%
  \renewcommand*\fixmenoteprefix{\fixmelogo~Anm}
  \renewcommand*\fixmewarningprefix{\fixmelogo~Warnung}
  \renewcommand*\fixmeerrorprefix{\fixmelogo~Fehler}
  \renewcommand*\fixmefatalprefix{\fixmelogo}
  \renewcommand*\fixmeindexname{\fixmelogo}
  \renewcommand*\fixmenoteindexname{**a@Anmerkungen:!}
  \renewcommand*\fixmewarningindexname{**b@Warnungen:!}
  \renewcommand*\fixmeerrorindexname{**c@Fehler:!}
  \renewcommand*\fixmefatalindexname{}
  \renewcommand*{\listfixmename}{Verzeichnis der Korrekturen}}
\DeclareOption{ngerman}{%
  \ExecuteOptions{german}}

%    \end{macrocode}
%
% \subsection{Document status processing}
% \begin{macro}{\fixme@note}
%   \begin{macro}{\fixme@warning}
%     \begin{macro}{\fixme@error}
%       Here's where the draft or final version of the internal macros are
%       selected.
%     \end{macro}
%   \end{macro}
% \end{macro}
%    \begin{macrocode}
\DeclareOption{final}{
  \let\fixme@note\fixme@note@final
  \let\fixme@warning\fixme@warning@final
  \let\fixme@error\fixme@error@final
  \let\fixme@fatal\fixme@fatal@final
  \let\listoffixmes\listoffixmes@final
  }

\DeclareOption{draft}{
  \let\fixme@note\fixme@note@draft
  \let\fixme@warning\fixme@warning@draft
  \let\fixme@error\fixme@error@draft
  \let\fixme@fatal\fixme@fatal@draft
  \let\listoffixmes\listoffixmes@draft
  \@ifclassloaded{article}{
    \let\@listoffixmes@pretoc\@listoffixmes@pretoc@article
    \let\@listoffixmes@posttoc\@listoffixmes@posttoc@article
    \let\l@fixme\l@fixme@article}{
    \@ifclassloaded{report}{
      \let\@listoffixmes@pretoc\@listoffixmes@pretoc@report
      \let\@listoffixmes@posttoc\@listoffixmes@posttoc@report
      \let\l@fixme\l@fixme@report}{
      \@ifclassloaded{book}{
        \let\@listoffixmes@pretoc\@listoffixmes@pretoc@book
        \let\@listoffixmes@posttoc\@listoffixmes@posttoc@book
        \let\l@fixme\l@fixme@book}{
        \@ifclassloaded{scrartcl}{
          \let\@listoffixmes@pretoc\@listoffixmes@pretoc@scrartcl
          \let\@listoffixmes@posttoc\@listoffixmes@posttoc@scrartcl
          \let\l@fixme\l@fixme@scrartcl}{
          \@ifclassloaded{scrreprt}{
            \let\@listoffixmes@pretoc\@listoffixmes@pretoc@scrreprt
            \let\@listoffixmes@posttoc\@listoffixmes@posttoc@scrreprt
            \let\l@fixme\l@fixme@scrreprt}{
            \@ifclassloaded{scrbook}{
              \let\@listoffixmes@pretoc\@listoffixmes@pretoc@scrbook
              \let\@listoffixmes@posttoc\@listoffixmes@posttoc@scrbook
              \let\l@fixme\l@fixme@scrbook}{
              %% Use the article layout by default.
              \let\@listoffixmes@pretoc\@listoffixmes@pretoc@article
              \let\@listoffixmes@posttoc\@listoffixmes@posttoc@article
              \let\l@fixme\l@fixme@article}
            }
          }
        }
      }
    }
  }

%    \end{macrocode}
% Put us in \texttt{english} and \texttt{final} mode, and enable the marginal
% notes.
%    \begin{macrocode}
\ExecuteOptions{english,final,margin}
\ProcessOptions*

%    \end{macrocode}
% Finally, ensure backward compatibility for |\fixme| with |\FiXmeUser|, and
% output a summary giving the number of fixme notes at the end of the
% compilation:
%    \begin{macrocode}
\AtBeginDocument{%
  \ifx\FiXmeUser\empty
    \renewcommand\FiXmeUser[1]{\FXUser{\fixmefatalprefix}{#1}}
  \fi}

\AtEndDocument{%
  \GenericWarning{%
    (FiXme)\@spaces\@spaces\@spaces\@spaces
  }{%
    Package FiXme Summary: Number of notes: \thefixmenotecount,\MessageBreak%
    Number of warnings: \thefixmewarningcount,\MessageBreak%
    Number of errors: \thefixmeerrorcount,\MessageBreak%
    Number of fatal errors: \thefixmefatalcount,\MessageBreak%
    Total: \thefixmecount\@gobble%
  }}
%    \end{macrocode}
%
% ^^A \PrintChanges
% ^^A \PrintIndex
% \Finale
%
% ^^A fixme.dtx ends here.
